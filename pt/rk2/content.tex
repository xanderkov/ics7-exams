\section{Рубежный контроль 2}

\subsection{Сформулировать определение несовместных событий. Как связаны свойства несовместности и независимости событий?}

Определение. События 𝐴 и 𝐵 называются несовместными, если их произведение пусто . В противном случае события 𝐴 и 𝐵 называются совместными.
Определение. События 𝐴 и 𝐵 называются независимыми, если 𝑃(𝐴𝐵) = 𝑃(𝐴) 𝑃(𝐵).

\subsection{Сформулировать геометрическое определение вероятности.}

Геометрическое определение вероятности является обобщением классического определения на случай, когда |Ω| = ∞. 
Пусть 
1. Ω ⊆ $R^𝑛$ ;
2. 𝜇(Ω) < ∞, где 𝜇 — некая мера. Если 𝑛 = 1, то 𝜇 — это длина; если 𝑛 = 2, то 𝜇 — площадь; если 𝑛 = 3 — объём. Можно определить меры и при больших 𝑛; 
3. Возможность принадлежности некоторого элементарного исхода случайного эксперимента событию 𝐴 ⊆ 𝑄 пропорциональна мере этого события и не зависит от формы события 𝐴 и его расположения внутри Ω. 
Тогда 
Определение. Вероятностью случайного события 𝐴 ⊆ Ω называют число 𝑃{𝐴} = 𝜇(𝐴)/ 𝜇(Ω)

\subsection{Сформулировать определение сигма-алгебры событий. Сформулировать ее основные свойства.}

Для строгого аксиоматического определения вероятности необходимо уточнить понятие события:
1. Данное выше определение события как произвольного подмножества множества Ω в случае бесконечного множества Ω приводит к противоречивой теории (см. парадокс Рассела); 
2. Таким образом, необходимо в качестве события рассматривать не все возможные подмножества множества Ω, а лишь некоторые из них; 
3. Набор подмножеств множества Ω, выбранных в качестве событий, должен обладать рядом свойств. Понятно, что если 𝐴 и 𝐵 — связанные со случайным экспериментом события и известно, что в результате эксперимента они произошли (или не произошли), то естественно знать, произошли ли события 𝐴 + 𝐵, 𝐴 · 𝐵, 𝐴,  . . .
Эти соображения приводят к следующему определению.
Пусть 
1. Ω — пространство элементарных исходов, связанных с некоторым случайным экспериментом; 
2. 𝛽 ̸= ∅ — система (набор) подмножеств в множестве Ω.
Определение. 
𝛽 называется сигма-алгеброй событий, если выполнены условия: 
1. Если 𝐴 ∈ 𝛽, то -𝐴 ∈ 𝛽;
2. Если 𝐴1, . . . , 𝐴𝑛, . . . ∈ 𝛽, то 𝐴1 + . . . + 𝐴𝑛 + . . . ∈ 𝛽.


\subsection{Сформулировать аксиоматическое определение вероятности. Сформулировать основные свойства вероятности.}

Пусть 
1. Ω — пространство элементарных исходов некоторого случайного эксперимента; 
2. 𝛽 — сигма-алгебра, заданная на Ω. 
Определение. Вероятностью (вероятностной мерой) называется функция 𝑃 : 𝛽 → R
1. ∀𝐴 ∈ 𝛽 =⇒ 𝑃(𝐴) >= 0 (аксиома неотрицательности);
2. 𝑃(Ω) = 1 (аксиома нормированности); 
3. Если 𝐴1, . . . , 𝐴𝑛, . . . — попарно несовместные события, то вероятность осуществления их суммы равна сумме вероятностей осуществления каждого из них по отдельности: 𝑃(𝐴1 + . . . + 𝐴𝑛 + . . .) = 𝑃(𝐴1) + . . . + 𝑃(𝐴𝑛) + . . . (расширенная аксиома сложения).

\subsection{Записать аксиому сложения вероятностей, расширенную аксиому сложения вероятностей и аксиому непрерывности вероятности. Как они связаны между собой?}

Сложение – Для любого конечного набора попарно несовместных событий 𝐴1, . . . , 𝐴𝑛 вероятность осуществления их суммы равна сумме вероятностей осуществления каждого из них по отдельности.
Расширенная – Если 𝐴1, . . . , 𝐴𝑛, . . . — попарно несовместные события, то вероятность осуществления их суммы равна сумме вероятностей осуществления каждого из них по отдельности: 𝑃(𝐴1 + . . . + 𝐴𝑛 + . . .) = 𝑃(𝐴1) + . . . + 𝑃(𝐴𝑛) + . . . (расширенная аксиома сложения).
Непрерывность –

\subsection{Сформулировать определение условной вероятности и ее основные свойства.}

Пусть 
1. 𝐴 и 𝐵 — два события, связанные с одним случайным экспериментом; 
2. Дополнительно известно, что в результате эксперимента произошло событие 𝐵
Условной вероятностью осуществления события 𝐴 при условии, что произошло 𝐵, называется число 

Свойства идентичны свойствам обычной (безусловной) вероятности.

\subsection{Сформулировать теоремы о формулах умножения вероятностей для двух событий и для произвольного числа событий.}

Теорема. Формула умножения вероятностей для двух событий 
Пусть 
1. 𝐴, 𝐵 — события; 
2. 𝑃(𝐴) > 0. 
Тогда 𝑃(𝐴𝐵) = 𝑃(𝐴) 𝑃(𝐵 | 𝐴)



\subsection{Сформулировать определение пары независимых событий. Как независимость двух событий связана с условными вероятностями их осуществления?}

Пусть 𝐴 и 𝐵 — два события, связанные с некоторым случайным экспериментом.
Определение. События 𝐴 и 𝐵 называются независимыми, если 𝑃(𝐴𝐵) = 𝑃(𝐴) 𝑃(𝐵).

Замечание. Разумеется, в качестве определения независимых событий логично было бы использовать условия 𝑃(𝐴 | 𝐵) = 𝑃(𝐴) или 𝑃(𝐵 | 𝐴) = 𝑃(𝐵) (6) Однако эти условия имеют смысл лишь тогда, когда 𝑃(𝐴) или 𝑃(𝐵) отлично от нуля. Условие же 𝑃(𝐴𝐵) = 𝑃(𝐴)𝑃(𝐵) «работает» всегда без ограничений.

\subsection{Сформулировать определение попарно независимых событий и событий, независимых в совокупности. Как эти свойства связаны между собой?}

Определение. События 𝐴1, . . . , 𝐴𝑛 называется попарно независимыми, если
∀∀ 𝑖 ̸= 𝑗; 𝑖, 𝑗 ∈ {1, . . . , 𝑛} 𝑃{𝐴𝑖𝐴𝑗} = 𝑃{𝐴𝑖}𝑃{𝐴𝑗}
Определение. События 𝐴1, . . . , 𝐴𝑛 называются независимыми в совокупности, если 
∀ 𝑘 ∈ {2, . . . , 𝑛} ∀∀ 𝑖1 < 𝑖2 < . . . < 𝑖𝑘 𝑃{𝐴𝑖1 , . . . , 𝐴𝑖𝑘 } = 𝑃{𝐴𝑖1 } · . . . · 𝑃{𝐴𝑖𝑘 }
1 <- 2


\subsection{Сформулировать определение полной группы событий. Верно ли, что некоторые события из полной группы могут быть независимыми?}

Пусть Ω — пространство элементарных исходов, связанных с некоторым случайным экспериментом, а (Ω, 𝛽, 𝑃) — вероятностное пространство этого случайного эксперимента. Определение. Говорят,  что события 𝐻1, . . . , 𝐻𝑛 ∈ 𝛽 образуют полную группу событий, если
1. 𝑃(𝐻𝑖) > 0, 𝑖 = 1, 𝑛; 
2. 𝐻𝑖𝐻𝑗 = ∅ при 𝑖 ̸= 𝑗; 
3. 𝐻1 + . . . + 𝐻𝑛 = Ω.
Да, верно.

\subsection{Сформулировать теорему о формуле полной вероятности.}

Теорема. Формула полной вероятности. Пусть 
1. 𝐻1, . . . , 𝐻𝑛 — полная группа событий;
2. 𝐴 ∈ 𝛽 — событие.
Тогда (это выражение называется формулой полной вероятности): 𝑃(𝐴) = 𝑃(𝐴 | 𝐻1)𝑃(𝐻1) + . . . + 𝑃(𝐴 | 𝐻𝑛)𝑃(𝐻𝑛)

\subsection{Сформулировать теорему о формуле Байеса.}

Теорема. Пусть 
1. 𝐻1, . . . , 𝐻𝑛 — полная группа событий; 
2. 𝑃(𝐴) > 0. 

\subsection{Дать определение схемы испытаний Бернулли. Записать формулу для вычисления вероятности осуществления ровно k успехов в серии из n испытаний.}

Определение. 

\subsection{Сформулировать определение элементарного исхода случайного эксперимента и пространства элементарных исходов. Сформулировать классическое определение вероятности. Привести пример.}

Определение. Множество Ω всех исходов данного случайного эксперимента называют пространством элементарных исходов. 
1. Каждый элементарный исход является «неделимым», т. е. он не может быть разбит на более «мелкие» исходы; 
2. В результате каждого эксперимента обязательно имеет место ровно один из входящих в Ω элементарных исходов.

\subsection{Сформулировать классическое определение вероятности. Опираясь на него, доказать основные свойства вероятности}

Свойства вероятности:
1. Вероятность 𝑃(𝐴) > 0 (неотрицательна). 
2. 𝑃(Ω) = 1. 
3. Если 𝐴, 𝐵 — несовместные события, то 𝑃(𝐴 + 𝐵) = 𝑃(𝐴) + 𝑃(𝐵)
Доказательство:

\subsection{Сформулировать статистическое определение вероятности. Указать его основные недостатки.}

1. Некоторый случайный эксперимент произведён 𝑛 раз; 
2. При этом некоторое наблюдаемое в этом эксперименте событие 𝐴 произошло 𝑛𝐴 раз.

У статистического определения полным-полно недостатков:
1. Никакой эксперимент не может быть произведён бесконечное много раз;
2. С точки зрения современной математики статистическое определение является архаизмом, т. к. не даёт достаточно базы для дальнейшего построения теории.

\subsection{Сформулировать определение сигма-алгебры событий. Доказать ее основные свойства.}

Для строгого аксиоматического определения вероятности необходимо уточнить понятие события:
1. Данное выше определение события как произвольного подмножества множества Ω в случае бесконечного множества Ω приводит к противоречивой теории (см. парадокс Рассела); 
2. Таким образом, необходимо в качестве события рассматривать не все возможные подмножества множества Ω, а лишь некоторые из них;
3. Набор подмножеств множества Ω, выбранных в качестве событий, должен обладать рядом свойств. Понятно, что если 𝐴 и 𝐵 — связанные со случайным экспериментом события и известно, что в результате эксперимента они произошли (или не произошли), то естественно знать, произошли ли события 𝐴 + 𝐵, 𝐴 · 𝐵, -𝐴,  . . .
Эти соображения приводят к следующему определению. Пусть 
1. Ω — пространство элементарных исходов, связанных с некоторым случайным экспериментом; 
2. 𝛽 ̸= ∅ — система (набор) подмножеств в множестве Ω.
Определение. 𝛽 называется сигма-алгеброй событий, если выполнены условия: 
1. Если 𝐴 ∈ 𝛽, то 𝐴 ∈ 𝛽; 
2. Если 𝐴1, . . . , 𝐴𝑛, . . . ∈ 𝛽, то 𝐴1 + . . . + 𝐴𝑛 + . . . ∈ 𝛽
Свойства:
1. Ω ∈ 𝛽; 
2. ∅ ∈ 𝛽; 
3. Если 𝐴1, . . . , 𝐴𝑛, . . . ∈ 𝛽, то 𝐴1 · . . . · 𝐴𝑛 · . . . ∈ 𝛽; 
4. Если 𝐴, 𝐵 ∈ 𝛽; то 𝐴 r 𝐵 ∈ 𝛽
Доказательства:

\subsection{Сформулировать аксиоматическое определение вероятности. Доказать свойства вероятности для дополнения события, для невозможного события, для следствия события.}

усть 
1. Ω — пространство элементарных исходов некоторого случайного эксперимента;
2. 𝛽 — сигма-алгебра, заданная на Ω.
Определение. Вероятностью (вероятностной мерой) называется функция
𝑃 : 𝛽 → R
1. ∀𝐴 ∈ 𝛽 =⇒ 𝑃(𝐴) > 0 (аксиома неотрицательности); 
2. 𝑃(Ω) = 1 (аксиома нормированности); 
3. Если 𝐴1, . . . , 𝐴𝑛, . . . — попарно несовместные события, то вероятность осуществления их суммы равна сумме вероятностей осуществления каждого из них по отдельности: 𝑃(𝐴1 + . . . + 𝐴𝑛 + . . .) = 𝑃(𝐴1) + . . . + 𝑃(𝐴𝑛) + . . . (расширенная аксиома сложения).



Доказательства:

\subsection{Сформулировать аксиоматическое определение вероятности. Сформулировать свойства вероятности для суммы двух событий и для суммы произвольного числа событий. Доказать первое из этих свойств.}

Пусть 
1. Ω — пространство элементарных исходов некоторого случайного эксперимента;
2. 𝛽 — сигма-алгебра, заданная на Ω.
Определение. Вероятностью (вероятностной мерой) называется функция
𝑃 : 𝛽 → R
1. ∀𝐴 ∈ 𝛽 =⇒ 𝑃(𝐴) > 0 (аксиома неотрицательности); 
2. 𝑃(Ω) = 1 (аксиома нормированности); 
3. Если 𝐴1, . . . , 𝐴𝑛, . . . — попарно несовместные события, то вероятность осуществления их суммы равна сумме вероятностей осуществления каждого из них по отдельности: 𝑃(𝐴1 + . . . + 𝐴𝑛 + . . .) = 𝑃(𝐴1) + . . . + 𝑃(𝐴𝑛) + . . . (расширенная аксиома сложения).

Формулировка:

\subsection{Сформулировать определение условной вероятности. Доказать, что она удовлетворяет трем основным свойствам безусловной вероятности.}

Пусть
1. 𝐴 и 𝐵 — два события, связанные с одним случайным экспериментом;
2. Дополнительно известно, что в результате эксперимента произошло событие 𝐵.
Определение. Условной вероятностью осуществления события 𝐴 при условии, что произошло 𝐵, называется число

Теорема:
Пусть 
1. Зафиксировано событие 𝐵, 𝑃(𝐵) ̸= 0; 
2. 𝑃(𝐴 | 𝐵) рассматривается как функция события 𝐴. 
Тогда 𝑃(𝐴 | 𝐵) обладает всеми свойствами безусловной вероятности.
Доказательство:

\subsection{Доказать теоремы о формулах умножения вероятностей для двух событий и для произвольного числа событий.}
\subsection{Сформулировать определение пары независимых событий. Сформулировать и доказать теорему о связи независимости двух событий с условными вероятностями их осуществления.}

Пусть 𝐴 и 𝐵 — два события, связанные с некоторым случайным экспериментом. 
Определение. События 𝐴 и 𝐵 называются независимыми, если 𝑃(𝐴𝐵) = 𝑃(𝐴) 𝑃(𝐵).
Теорема. . . . 
1. Пусть 𝑃(𝐵) > 0. Утверждение «𝐴 и 𝐵 — независимы» равносильно 𝑃(𝐴 | 𝐵) = 𝑃(𝐴); 
2. Пусть 𝑃(𝐴) > 0. Утверждение «𝐴 и 𝐵 — независимы» равносильно 𝑃(𝐵 | 𝐴) = 𝑃(𝐵).

\subsection{Сформулировать определение попарно независимых событий и событий, независимых в совокупности. Показать на примере, что из первого не следует второе.}

Определение. События 𝐴1, . . . , 𝐴𝑛 называется попарно независимыми, если
∀∀ 𝑖 ̸= 𝑗; 𝑖, 𝑗 ∈ {1, . . . , 𝑛} 𝑃{𝐴𝑖𝐴𝑗} = 𝑃{𝐴𝑖}𝑃{𝐴𝑗}
Определение. События 𝐴1, . . . , 𝐴𝑛 называются независимыми в совокупности, если
∀ 𝑘 ∈ {2, . . . , 𝑛} ∀∀ 𝑖1 < 𝑖2 < . . . < 𝑖𝑘 𝑃{𝐴𝑖1 , . . . , 𝐴𝑖𝑘 } = 𝑃{𝐴𝑖1 } · . . . · 𝑃{𝐴𝑖𝑘 }
Пример. (Бернштейна) 
Рассмотрим правильный тетраэдр, на одной грани которого «написано» 1, второй — 2, третьей — 3, четвёртой — 1, 2, 3.
Этот тетраэдр один раз подбрасывают. 
Событие 𝐴1 заключается в том, что на нижней грани «написано» 1; также введём 𝐴2 для 2, 𝐴3 для 3. Давайте покажем, что события 𝐴1, 𝐴2, 𝐴3 попарно независимы, но не являются независимыми в совокупности. 
1. Докажем, что они независимы попарно. Т. к. 𝑃(𝐴1) = 1 2 , 𝑃(𝐴2) = 1 2 , то
𝑃(𝐴1𝐴2) = 𝑃(𝐴1) 𝑃(𝐴2) = ¼
Событие 𝐴1𝐴2 означает, что на нижней грани присутствуют и 1, и 2. Всё аналогично для 𝑃(𝐴1𝐴3) = 𝑃(𝐴1)𝑃(𝐴3) и 𝑃(𝐴2𝐴3) = 𝑃(𝐴2)𝑃(𝐴3).
2. Проверим равенство 𝑃(𝐴1𝐴2𝐴3) = 𝑃(𝐴1) 𝑃(𝐴2) 𝑃(𝐴3), которое, казалось бы, должно равняться 1/8 . Но произведение событий 𝐴1, 𝐴2, 𝐴3 означает, что на нижней грани присутствуют и 1, и 2, и 3, вероятность чего равна 1/4 . И выходит, что 1/4 ̸= 1/8 .
Следовательно, события 𝐴1, 𝐴2, 𝐴3 не являются независимыми в совокупности.

\subsection{Доказать теорему о формуле полной вероятности.}

Пусть Ω — пространство элементарных исходов, связанных с некоторым случайным экспериментом, а (Ω, 𝛽, 𝑃) — вероятностное пространство этого случайного эксперимента.
Определение. Говорят. что события 𝐻1, . . . , 𝐻𝑛 ∈ 𝛽 образуют полную группу событий, если 
1. 𝑃(𝐻𝑖) > 0, 𝑖 = 1, 𝑛;
2. 𝐻𝑖𝐻𝑗 = ∅ при 𝑖 ̸= 𝑗; 
3. 𝐻1 + . . . + 𝐻𝑛 = Ω.
Теорема. Формула полной вероятности. Пусть 
1. 𝐻1, . . . , 𝐻𝑛 — полная группа событий; 
2. 𝐴 ∈ 𝛽 — событие. 
Тогда (это выражение называется формулой полной вероятности):
𝑃(𝐴) = 𝑃(𝐴 | 𝐻1)𝑃(𝐻1) + . . . + 𝑃(𝐴 | 𝐻𝑛)𝑃(𝐻𝑛)
Доказательство:

\subsection{Доказать теорему о формуле Байеса.}

Теорема

\subsection{Доказать формулу для вычисления вероятности осуществления ровно k успехов в серии из n испытаний по схеме Бернулли..}
