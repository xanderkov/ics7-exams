\section{Рубежный контроль 3}

\subsection{Сформулировать определения случайной величины и функции распределения вероятностей случайной величины. Записать основные свойства функции распределения.}

Случайной величиной естественно называть числовую величину, значение которой зависит от того, какой именно \textbf{элементарный исход} произошел в результате эксперимента со случайным исходом.
Множество всех значений, которые случайная величина может принимать, называют \textbf{множеством возможных значений} этой \textbf{случайной величины.}

\textbf{Определение}

Пусть ($\Omega$, $\beta$, P) --- вероятностное пространство.

Случайной величиной называется функция $X : \Omega \rightarrow R$
Такая, что $\forall x \in \Re$ множество $(\omega : X (\omega) < x) \in \beta$.

\textbf{Определение}

Функцией распределения вероятностной случайной величины X называется отображение $F_X : R \rightarrow R$, определяется правилом $F_X(x) = P\{X < x\}$

\textbf{Свойства}

\begin{enumerate}[label=\arabic*.]
	\item $\lim\limits_{x \rightarrow -\infty}F_X(x) = 0, \lim\limits_{x \rightarrow +\infty}F_X(x) = 1$;
	\item $0 \leq F(x) \leq 1$;
	\item $F(x_1) \leq F(x_2)$, при $x_1 < x_2$ ($F_x$ --- не убывающая функция);
	\item $P\{a \leq X < b\} = F_X(b) - F_X(a)$;
	\item $\lim\limits_{x \rightarrow x_0}F_x(x) = F_X(x_0)$ --- функция распределения непрерывна слева в каждой точке $x_0$.
\end{enumerate}

\subsection{Сформулировать определение дискретной случайной величины; понятие ряда распределения.Сформулировать определение непрерывной случайной величины и функции плотности распределения вероятностей.}

\textbf{Определение}: Дискретная

Случайная величина называется дискретной, если множество ее возможных значений конечно или счетно.

Если множество значений дискретной случайной величины конечно, то закон ее распределения можно задать с использованием таблицы.
\begin{table}[ht!]
	\begin{center}
		\caption{Все воможные значения СВ X $p_i = P\{X=x_i\}, i = \overline{1,n}$}
		\label{tbl:best}
		\begin{tabular}{|c|c|c|c|c|}
			\hline
			X & $x_1$ & $x_2$ & $\dots$ & $x_n$ \\ 
			\hline
			P & $p_1$ & $p_2$ & $\dots$ & $p_n$ \\
			\hline
		\end{tabular}
	\end{center}
\end{table}

\textbf{Определения}

Таблицу \ref{tbl:best} называется рядом распределения дискретной случайной величины.

\textbf{Определение}

Случайной величиной X называется непрерывной, если $\exists f: \Re \rightarrow \Re: \forall x \in \Re$ значение $F_X(x)$ можно представить в виде:\
$F_X(x) = \int_{\infty}^{\infty} f_X(t)dt$ При этом f называется функцией плотностью распределения случайной величины X.

\subsection{Сформулировать определение непрерывной случайной величины. Записать основные свойства функции плотности распределения вероятностей непрерывной случайной величины.}

\textbf{Свойства}

\begin{enumerate}[label=\arabic*.]
	\item f(x) $\leq$ 0, $x \in \Re$;
	\item $P\{x_1 \leq X < x_2\} = \int_{x_1}^{x_2}f(x)dx$;
	\item $\int_{\infty}f(x)dx = 1$ - условие нормировки;
	\item $P\{x_0 \leq X < x_0 + \Delta x\} \eqsim f(x_0)\Delta x$, где $\Delta x$ --- мало, а f --- непрерывна а точке $x_0$;
	\item Если X --- непрерывна СВ, то для $\forall$ наперед заданной точке $x_0$ $P\{X=x_0\}=0$.
\end{enumerate}


\subsection{Сформулировать определения случайного вектора и его функции распределения вероятностей. Записать свойства функции распределения двумерного случайного вектора.}

\textbf{Определение}

n - мерным случайным вектором называется кортеж $(X_1, \dots, X_n)$, где $x_i, i=\overline{1,n}$ --- СВ, заданные на одном вероятностном пространстве.

\textbf{Определение}

Функцией распределения случайная вектора

$\overrightarrow{X} = (x_1, \dots x_n)$ называется отображение $F: \Re^n \rightarrow \Re$, определенной правилом $F(x_1, \dots, x_n) = P\{X_1 \leq x_1\ \dots, X_n < x_n\}$.

\textbf{Свойства}

\begin{enumerate}[label=\arabic*.]
	\item $0 \leq F(x_1, x_2) \leq 1$
	\item \begin{itemize}
		\item при фиксированном $x_2$ функция $F(x_1, x_2)$, как функция переменная $x_1$ является неубывающей
		\item аналогично $x_1$
	\end{itemize}
	\item $\lim\limits_{x_1 \rightarrow -\infty}F_x(x_1, x_2) = 0$, $\lim\limits_{x_2 \rightarrow -\infty}F_x(x_1, x_2) = 0$;
	\item $\lim\limits_{x_2 \rightarrow +\infty, x_1 \rightarrow +\infty}F(x_1, x_2) = 1$
	\item $\lim\limits_{x_2 \rightarrow +\infty, x_1 = const}F(x_1, x_2) = F_{X_1}(x_1)$, $\lim\limits_{x_1 \rightarrow +\infty, x_2 = const}F(x_1, x_2) = F_{X_2}(x_2)$
	\item $P\{a_1 \leq X_1\ < b_1, a_2 \leq X_2 < b_2\} = F(a_1, a_2) + F(b_1, b_2) - F(a_1, b_2) - F(b_1, a_2)$
	\item При фиксированном $x_2$ функция $F(x_1, x_2)$ как функция переменной $x_1$ является непрерывной слева во векторе. Аналогично $x_1$
\end{enumerate}

\subsection{Сформулировать определение дискретного случайного вектора; понятие таблицы распределения двумерного случайного вектора. Сформулировать определения непрерывного случайного вектора и его функции плотности распределения вероятностей.}

\textbf{Определение}

Случайный вектор $\overrightarrow{X} = (X_1, \dots, X_n)$ называется дискертным, если каждая из случайных величин $X_i, i=\overline{1,n}$ является дискретной.


\textbf{Таблица распределения}

Рассмотрим случай n=2

Пусть: 
\begin{enumerate}[label=\arabic*.]
	\item (X, Y) - двумерный вектор --- дискретный;
	\item будем считать, что X и Y принимают конечное множество значений.  
\end{enumerate}
$X \in \{x_1, \dots, x_n\}, Y \in \{y_1, \dots, y_n\}$

Это означает, что случайны вектор (X, Y) может принимать значения $(x_i, y_i)$, $i=\overline{1,m}, j=\overline{1,n}$. Закон распределения такого вектора часто задают таблицей:
 
\begin{table}[ht!]
	\begin{center}
		\caption{$p_{ij} = P\{(X, Y)=(x_i, y_j)\} = P\{\{X=x_i\} \cdot \{Y=y_j\}\} = P\{X=x_i, Y=y_j\}, $ при этом достаточно выполняется условие нормировки $\sum_{m}^{i=1}\sum_{n}^{j=1}p_{ij}=1$;}
		\label{tbl:best}
		\begin{tabular}{|c|c|c|c|c|c|c|}
			\hline
			$XY$ & $y_1$ & $\dots$ & $y_j$ & $\dots$ &  $y_n$ & $P_X$\\  \hline
			$x_1$ & $p_{11}$ & $\dots$ & $p_{1j}$ & $\dots$ &  $p_{1n}$ & $p_{x1}$\\  \hline
			$\dots$ & $\dots$ & $\dots$ & $\dots$ & $\dots$ &  $\dots$ & $\dots$\\  \hline
			$x_i$ & $p_{i1}$ & $\dots$ & $p_{ij}$ & $\dots$ &  $p_{in}$ & $p_{xi}$\\  \hline
			$\dots$ & $\dots$ & $\dots$ & $\dots$ & $\dots$ &  $\dots$ & $\dots$\\ \hline
			$x_m$ & $p_{m1}$ & $\dots$ & $p_{mj}$ & $\dots$ &  $p_{mn}$ & $p_{xm}$\\  \hline
			$p_y$ & $p_{y1}$ & $\dots$ & $p_{yj}$ & $\dots$ &  $p_{yn}$ & $1$\\  
			\hline
		\end{tabular}
	\end{center}
\end{table}

\textbf{Определение}

Случайный вектор $(X_1, \dots X_n)$ называется непрерывным, если его функцию распределения можно представить в виде: 
\begin{equation}
	F_X(x_1, \dots, x_n) = \int_{-\infty}^{x_1}dt_1\int_{-\infty}^{x_2}dt_2 \dots
	\int_{-\infty}^{xn1}dt_n.
\end{equation}

При том функция $f$ называется функцией плотности распределения вероятностей случайного вектора $(X_1, \dots X_n)$.

\subsection{Сформулировать определения непрерывного случайного вектора и его функции плотности распределения вероятностей. Записать основные свойства функции плотности распределения двумерных случайных векторов.}

\textbf{Свойства} для n = 2

\begin{enumerate}[label=\arabic*.]
	\item $f(x_1, x_2) \leq 0$;
	\item $P\{a_1 \leq X_1 \leq b_1, a_2 \leq X_2 \leq b_2\} = \int_{a_1}^{b_1}dx_1\int_{a_2}^{b_2}f(x_1, x_2)dx_2$;
	\item $\int\int\limits_{R_1} f(x_1, x_2)dx_1dx_2 = 1$;
	\item $P\{x_1 \leq X_1 < x_1 + \Delta x_1, x_2 \leq X_2 < x_2 + \Delta x_2\} \eqsim f(x_1, x_2) \Delta x_1 \Delta x_2$, если $(x_1, x_2)$ - точка непрерывности функции f;
	\item Если $(X_1, X_2)$ --- непр. случайные вектор, то для $\forall$ наперед заданного $(x_1^0, x_2^0), P\{(X_1, X_2 = (x_1^0, x_2^0))\} = 0$;
	\item $P\{(X_1, X_2) \in \DJ\} \int\int\limits_{\DJ}f(x_1, x_2)dx_1dx_2$;
	\item $\int_{\infty}^{\infty}f(x_1, x_2)dx_2 = f_{x_1}(x_1) \int_{\infty}^{\infty}f(x_1, x_2)dx_1 = f_{x_2}(x_2)$
\end{enumerate}

\subsection{Сформулировать определение независимых случайных величин. Сформулировать свойства независимых случайных величин. Сформулировать определение попарно независимых случайных величин и случайных величин, независимых в совокупности.}

\textbf{Это не надо, но мне жалко это выкидывать}

Пусть 
\begin{enumerate}
	\item (X, Y) --- дискретный случайны вектор множество значений, которого конечно;
	\item причем $X \in \{x_1, \dots x_m\} Y \in \{y_1, \dots y_n\}$;
	\item $x_1 < \dots < x_n y_1 < \dots < y_n$.
\end{enumerate}

В случае такого случайного вектора (X, Y) определение независимых случайных величин по аналогии с определением событий можно сформулировать так:

X, Y называют независимыми., если $P\{(X, Y)=(x_i, y_j)\} = P\{X=x_i\} \cdot P\{Y=y_i\}; \{P\{\{X=x_i\}\{Y=y_i\}\}\},$ $i=\overline{1,m}, j=\overline{1,n}$ 

\textbf{Это надо}

\textbf{Определение}

Случайный величины X и Y называются независимыми, если $F(x, y) = F_X(x)F_Y(y)$, где F --- совместная функция распределения случайных величин X и Y.

$F_X, F_Y$ --- маргинальная функция распределения случайных величин X и Y.

\textbf{Свойства}

\begin{enumerate}[label=\arabic*.]
	\item Случайные величины X и Y незав. $\Leftrightarrow$ для $\forall x \in \Re, \forall y \in \Re$ события $\{X<x\}$ и $\{Y < y\}$ независимы.
	\item Случайные величины X, Y независимы $\Leftrightarrow$ $\forall \forall x_1, x_2 \in \Re$ $\forall \forall y_1, y_2 \in \Re$ события $\{x_1 \leq X < x_2\}$ и $\{y_1 \leq Y < y_2\}$ независимы.
	\item Случайные величины X и Y независимы $\Leftrightarrow$ $\forall M_1$ и $\forall M_2$ события $\{X \in M_1\}$ и $\{Y \in M_2\}$ независимы, где $M_1, M_2$ --- промежутки или обозначения промежутков в $\Re$
	\item Если \begin{enumerate}[label=\arabic*]
		\item X и Y дискретная случайная величина;
		\item $X \in \{x_1, \dots, x_n\}$, $Y \in \{y_1, \dots, y_n\}$;
		\item $P\{(X, Y)=(x_i, y_j)\} = p_{ij}$, $P\{X=x_i\}=p_{xi}, P\{Y=y_i\}=p_{yi}$, $i=\overline{1,m}, j=\overline{1,n}$,
	\end{enumerate}
	то X, Y независимы $\Leftrightarrow$ $p_{ij} = p_{xi}p_{y_i}$ $i=\overline{1,m}, j=\overline{1,n}$
	\item Если X, Y непрерывные случайные величины, то X, Y независимы $\Leftrightarrow$ $f(x, y) = f_X(x)f_Y(y)$, где f --- совм. плотность распределения случайного вектора X и Y $f_X, f_Y$ --- ера плотности.
\end{enumerate}

\textbf{Определение}

Случайный величины $X_1, \dots, X_n$ заданные на одном и том же вероятностном пространстве, называют \textbf{независимыми в совокупности}, если $F(x_1, \dots, x_n) = F_{X_1}(x_1) \dot \cdot \dot F_{X_n}(x_n)$, где F --- совместная функция распределения случайных величин $X_i, \forall i,j = \overline{1,n}, i \neq j$.

\textbf{Определение}

Случайный величины $X_1, \dots, X_n$, заданные на одном и том же вероятностном пространстве, называют \textbf{независимыми в попарно}, если $x_i$ и $x_j$ независимы  для $\forall i,j = \overline{1,n}, i \neq j$.

\subsection{Понятие условного распределения. Доказать формулу для вычисления условного ряда распределения одной компоненты двумерного дискретного случайного вектора при условии, что другая компонента приняла определенное значение. Записать формулу для вычисления условной плотности распределения одной компоненты двумерного непрерывного случайного вектора при условии, что другая компонента приняла определенное значение.}

• Пусть дан двумерный СВектор (Х,У) и известно, что СВ У принимает значение у.

• Пусть (Х,У) – дискретный СВектор; . Пусть для некоторого j . Условной вероятностью того, что СВ Х примет значение xi при условии что У принимает значение yj, называется число ; набор вероятностей  называется условным распределением СВ Х.

• Пусть (ХУ) – непрерывный СВектор. Условной функцией распределения СВ Х при условии  называется отображение  Условной плотностью распределения СВ Х при условии У=у называется функция , где f(x,y) – совместная плотность распределения СВектора.

\subsection{Сформулировать определение независимых случайных величин. Сформулировать критерий независимости двух случайных величин в терминах условных распределений.}

• Пусть (Х,У) – двумерный случайный вектор. Тогда:

1. СВ Х,У независмы 

2. Если (Х,У) – НСВектор, то Х,У независимы 

3. Если (Х,У) – ДСВектор, то Х,У независимы 

\subsection{Понятие функции случайной величины. Указать способ построения ряда распределения функции дискретной случайной величины. Сформулировать теорему о плотности распределения функции от непрерывной случайной величины.}

• СВ У, которая каждому значению СВ Х ставит в соответствие число , называют скалярной функцией скалярной СВ Х. При этом сама У также является случайной величиной: если Х – ДСВ, то У – также ДСВ; если Х – НСВ, то У может быть НСВ, ДСВ или СВ смешаного типа.

• Если Х – ДСВ, то ряд распределения У строится следующим образом – в первой строке записываются значения , а во вторую строку переписываются значения .

• Теорема: если Х – НСВ с плотностью распределения ,   - монотонная и непрерывно диффернцируемая скалярная функция, а  – обратная к ), то для СВ  функция распределения .

\subsection{Понятие скалярной функции случайного векторного аргумента. Доказать формулу для нахождения значения функции распределения случайной величины Y , функционально зависящей от случайных величин X1 и X2 .}

• Пусть (Х1, Х2) – СВектор,  - скалярная функция. СВ  называют скалярной функией случайного вектора.

• Теорема: Пусть (Х1,Х2) – НСВектор и . Тогда 
Доказательство:  эквивалентны. Следовательно, .


\subsection{Сформулировать и доказать теорему о формуле свертки.}

• Теорема: пусть (Х,У) – СВектор, непрерывный и независимый, а 
Доказательство:  Т.к. Х,У независимы, то  следовательно 
Наконец, 
Выражение 



\subsection{Сформулировать определение математического ожидания случайной величины (дискретный и непрерывный случаи). Записать формулы для вычисления математического ожидания функции от случайной величины. Сформулировать свойства математического ожидания. Механический смысл математического ожидания.}


• ДСВ: Математическим ожиданием СВ Х называется число  пробегает множество всех значений Х.

НСВ: Математическим ожиданием СВ Х называется число , где f(x) – плотность распределения НСВ Х.
• Если Х – СВ,  - скалярная функция, то  для ДСВ и  для НСВ.

• Механический смысл мат.ожидания: пусть есть стержень, обладающий «вероятностной массой» и в xi лежит её pi часть. Тогда математическое ожидание задаёт x0 – центр тяжести для этого стержня. В случае НСВ, f(x) можно интерпретировать как «плотность» бесконечного стержня.
Свойства МО:
1) Если Х принимает значение х0 с вероятностью 1 (т.е. не является СВ), то MX=x0.
2) 
3) 
4) Если Х и У независимые, то 

\subsection{Сформулировать определение дисперсии случайной величины. Записать формулы для вычисления дисперсии в дискретном и непрерывном случае. Сформулировать свойства дисперсии. Механический смысл дисперсии.}

• Дисперсией СВ Х называют математическое ожидание квадрата отклонения СВ Х от её среднего значения: .
Для ДСВ: ;    для НСВ: 
Механический смысл. Дисперсия представляет собой второй момент центрированной СВ Х:  //коментарий автора: это не икс в нулевой, это икс с кружочком сверху
• Свойства дисперсии:

1) Если СВ Х принимает всего одно значени С с вероятностью 1, то DC = 0

2) 

3) 

4) , если Х и У – независимые СВ.

\subsection{Сформулировать определения начального и центрального моментов случайной величины. Математическое ожидание и дисперсия как моменты. Сформулировать определение квантили и медианы случайной величины.}

• Начальным моментом К-го порядка СВ Х называют математическое ожидание К-й степени этой СВ:  .

• Центральным моментом К-го порядка Х называют матожидание к-й степени величины .

• Математическое ожидание СВ Х – совпадает с моментом первого порядка. Дисперсия совпадает с центральным моментом 2-го порядка.

• Квантилью СВ Х уровня а называется число , определяемое соотношением . Медианой СВ Х называется её квантиль уровня 0.5.

\subsection{Сформулировать определение ковариации случайных величин. Записать формулы для вычисления ковариации в дискретном и непрерывном случаях. Сформулировать свойства ковариации.}

• Коварацией СВ Х и У называется число  где m1=MX, m2=MY.
Если Х,У – ДСВ, то ковариация ;  если НСВ - .

• Свойства ковариации:

1) 

2) , если Х,У – независимые СВ

3) 

4) 

5) Равенство  верно тогда и только тогда, когда СВ Х,У связаны линейной зависимостью, т.е. .

6) 
