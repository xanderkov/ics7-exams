\section{Экзамен}

\subsection{Базы данных и системы управления базами данных. Определения, основные функции и классификация}

Случайной величиной естественно называть числовую величину, значение 

\subsection{Семантическое моделирование данных}

\textbf{Определение}: Дискретная


\subsection{Реляционная модель данных: структурная, целостная, манипуляционная части. Реляционная алгебра. Исчисление кортежей}

\textbf{Свойства}


\subsection{Теория проектирования реляционных баз данных: функциональные зависимости, нормальные формы}

Пусть 


\subsection{Теория проектирования хранилищ данных. Основные принципы построения. ETL и ELT процессы}

\textbf{Определение}

\subsection{Транзакции. Определение, свойства и уровни изоляции транзакций. Неблагоприятные эффекты, вызванные параллельным выполнением транзакци , и способы их устранения. Управление транзакциями и способы обработки ошибок}

\textbf{Свойства} для n = 2


\subsection{Блокировки. Определение, свойства, иерархии, гранулярность и взаимоблокировки, алгоритмы обнаружения взаимоблокировок}

\textbf{Определение}


\subsection{Журнализация. Операции журнала транзакций и его логическая и физическая архитектуры. Модели восстановления. Метаданные}

\textbf{Нет вопроса}

\subsection{Безопасность и Аудит. Ключевые понятия и участники системы безопасности. Модели управления доступом}

\textbf{Этого вопроса нет}


\subsection{MPP системы. Распределенное и колоночное хранение. Распределенные вычисления, модель MapReduce. Обеспечение отказоустойчивости.}

Пусть

\subsection{In-Memory базы данных. Преимущества и недостатки. Примеры использования}

Учитывая равенство $P\{Y < y\} = 	F_Y(y)$, приходим к формуле \ref{jopa2}.

\subsection{Инструкции языка описания данных, инструкции языка обработки данных, инструкции безопасности, инструкции управления транзакциями}

Когда $X_1, X_2$ являются \textit{независимыми случайными величинами, то есть их двумерная плотность распределения}



\subsection{Объекты базы данных: функции, процедуры, триггеры и курсоры}


\subsection{Оптимизация запроса: индексы, партиционирование, сегментирование}

\textbf{Дисперсией} случайной величины $X$ называют число



\subsection{План запроса. Этапы выполнения запроса}

Пусть $X$ --- случайная величина.
